% \\\\\\\\\\\\\\\\\\\\\\\\\\\\\\\\\\\\\\\\\\\\\\\\\\\\\\\\\\\\
% Mass Spring System
% \\\\\\\\\\\\\\\\\\\\\\\\\\\\\\\\\\\\\\\\\\\\\\\\\\\\\\\\\\\\
\topic{4. Mass Spring System}
\lis{
  \item Several mass points 
  \item Connected to each other by springs 
  \item Springs expand and stretch, exerting force on the mass points 
  \item Very often used to simulate cloth 
}
\subsubtitle{Newton's Laws}
\lis{
  \item Newton's 2nd law: \equ{\vec{F} = m\vec{a}} tells you how to compute acceleration, given the force and mass 
  \item Newton's 3rd law: If object A exerts a force F on object B, then object B is at the same time exerting force -F on A. 
}
\subsubtitle{Single spring}
\lis{
  \item Obeys the Hook's law: \equ{F=k(x-x_0)}
  \item $x_0$ = rest length 
  \item k = spring elasticity (aka stiffness)
  \item For $x < x_0$, spring wants to extend 
  \item For $x > x_0$, spring wants to contract
}
\subsubtitle{Hook's law in 3D}
\lis{
  \item Assume A and B two mass points connected with a spring. 
  \item Let L be the vector pointing from B to A: $L = A - B$
  \item Let R be the spring rest length 
  \item Then, the elastic force exerted on A is: \equ{\vec{F} = -k_{Hook}(|\vec{L}|-R)\frac{\vec{L}}{|\vec{L}|}} where $\frac{\vec{L}}{|\vec{L}|}$ is unit vector 
}
\subsubtitle{Damping}
\lis{
  \item Springs are not completely elastic 
  \item They absorb some of the energy and tend to decrease the velocity of the mass points attached to them 
  \item Damping force depends on the velocity: \equ{\vec{F}=-k_d\vec{v}}
  \item $k_d$ = damping coefficient 
  \item $k_d$ different than $k_{Hook}$
}
\subsubtitle{Damping in 3D}
\lis{
  \item Assume A and B two mass points connected with a spring 
  \item Let L be the vector pointing from B to A: $L=A-B$
  \item Then, the damping force exerted on A is: \equ{\vec{F}=-k_d\frac{(\vec{v_A}-\vec{v_B})\cdot \vec{L}}{|\vec{L}|}\frac{\vec{L}}{|\vec{L}|}}
  \item Here $v_A$ and $v_B$ are veclocities of points A and B
  \item Damping force always opposes the motion 
}
\subtitle{A network of springs}
\lis{
  \item Every mass point connected to some other points by springs 
  \item Springs exert forces on mass points: Hook's force, Damping force 
  \item Other forces: External force field (Gravity, Electical or magnetic force field), Collision force
}
\subsubtitle{Structural springs}
\lis{
  \item Connect very node to its 6 direct neighbors 
  \item Node (i,j,k) connected to (i+1,j,k), (i-1,j,k), (i,j+1,k), (i,j-1,k), (i,j,k+1), (i,j,k-1), for surface nodes, some of these neighbors might not exists 
  \item Structural springs establish the basic structure of the jello cube
}
\subsubtitle{Shear springs}
\lis{
  \item Disallow excessive shearing
  \item Prevent the cube from distorting 
  \item Every node (i,j,k) connected to its diagonal neighbors 
  \item Shear spring resists stretching and thus prevents shearing
}
\subsubtitle{Bend springs} 
\lis{
  \item Prevent the cube from folding over 
  \item Every node connected to its second neighbor in every direction (6 connections per node, unless surface node) 
  \item Bend spring resists contracting and thus prevents bending
}
\subsubtitle{External force field}
\lis{
  \item If there is an external force field, add that force to the sum of all the forces on a mass point \equ{\vec{F}_{total} = \vec{F}_{Hook} + \vec{F}_{damping} + \vec{F}_{force field}}
  \item There is one such equation for every mass point and for every moment in time
}
\subsubtitle{Collision detection}
\lis{
  \item The movement of the jello cube is limited to a bounding box 
  \item Collision detection easy: Check all the vertices if any of them is ouside the box 
  \item Inclined plane: \equ{F(x,y,z)=ax+by+cz+d=0} Initially, all points on the same side of the plane, $F(x,y,z)>0$ on one side of the plane and $F(x,y,z)<0$ on the other. Can check all the vertices for this condition
}
\subsubtitle{Collision response}
\lis{
  \item When collision happens, must perform some action to prevent the object penetrating even deeper 
  \item Object should bounce away from the colliding object 
  \item Some energy is usually lost during the collision 
  \item Several ways to handle collision response 
  \item We will use the penalty method
}
\subsubtitle{The penalty method}
\lis{
  \item When collision happens, put an artificial collision spring at the point of collision, which will push from the colliding object 
  \item Collision springs have elasticity and damping, just like ordinary springs 
  \item Direction is normal to the contact surface 
  \item Magnitude is proportional to the amount of penetration 
  \item Collision spring rest length is zero
}
\subsubtitle{Integrators}
\lis{
  \item Network of mass points and springs 
  \item Hook's law, damping law and Newton's 2nd law give acceleration of every mass point at any given time
  \item $F=ma$: Hook's law and damping provide $F$, $m$ is point mass, the value for $a$ follows from $F=ma$
  \item Now, we know acceleration at any given time for any point 
  \item Want to compute the actuall motion 
  \item The equations of motion: \equ{\frac{d \vec{x}}{dt}=\vec{v}}, \equ{\frac{d^2\vec{x}}{dt^2}=\frac{d\vec{v}}{dt}=\vec{a}(t)=\frac{1}{m}(\vec{F}_{Hook} + \vec{F}_{damping} + \vec{F}_{force field})}
  \item x = point position, v = point velocity, a = point acceleration 
  \item They describe the movement of any single mass point 
  \item $F_{hook}$ = sum of all Hook forces on a mass point 
  \item $F_{damping}$ = sum of all damping forces on a mass point 
  \item When we put these equations together for all the mass points, we obtain a system of ordinary differential equations 
  \item In general, impossible to solve analytically 
  \item Must solve numerically 
  \item Methods to solve such systems numerically are called integrators 
  \item Most widely used: Euler, Runge-Kutta 2nd order (aka the midpoint method) (RK2), Runge-Kutta 4th order (RK4)
  \item RK4 is often the method od choice 
  \item RK4 very popular for engineering applications 
  \item The time step should be inversely proportional to the square root of the elasticity k [Courant condition]
}
\subsubtitle{Integrator design issues}
\lis{
  \item Numerical stability: If time step too big, method "explodes"; t = 0.001 is a good starting choice for the assignment; Euler much more unstable than RK2 or RK4 (Requires smaller time-step, but is simple and hence fast); Euler rarely used in practice 
  \item Numerical accuracy: Smaller time steps means more stability and accuracy, but also means more computation 
  \item Computational cost: Tradeoff: accuray vs computation time
}

% \\\\\\\\\\\\\\\\\\\\\\\\\\\\\\\\\\\\\\\\\\\\\\\\\\\\\\\\\\\\
% FEM
% \\\\\\\\\\\\\\\\\\\\\\\\\\\\\\\\\\\\\\\\\\\\\\\\\\\\\\\\\\\\
\topic{The Finite Element Method} \\
\subsubtitle{Mass Spirng system vs FEM:}
\lis{
  \item mass spring system cannot capture volumetric effects
  \item mass spring system's behavior depends on the structure of the mesh
}
\subsubtitle{Hooke's law:}
\equ{P = \frac{f_n}{A} = E \frac{\Delta l}{l}}
\lis{
  \item \(P\): pressure, \(f\): force, \(A\): area, \(E\): Young's modulus
  \item the stronger \( \frac{f_n}{A}\), the larger the relative elongation \( \frac{\Delta l}{l}\)
  \item the magnitude of \( \frac{f_n}{A} \) increases when \( E \) increases
  \item \equ{\sigma = E \varepsilon} where \( \sigma = \frac{f_n}{A}\) is the applied stress, \( \varepsilon = \frac{\Delta l }{l}\) is the resulting strain
}
\notes{
  Graphics Interface (GI): a top conference in Canada \\
  Matthias Muller: \\
  https://matthias-research.github.io/pages/publications/GI2004.pdf \\ 
  Vega FEM: \\
  https://viterbi-web.usc.edu/~jbarbic/vega/ \\
  Inversion Recovery
}

\subsubtitle{Young's modulus:}
\equ{E}
\vspace{-0.7em}
\begin{itemize}
    \item describe the stiffness \chinese{刚度越低,柔度越高}
    \item \chinese{受到的力(应力)与所产生形变量(应变)之间的比值}
    \item Jello: \(10 Pa\)
    \item Brain: \(100 Pa\)
    \item Fat: \(1000Pa\)
    \item Muscle: \(10000Pa\)
    \item Human bone: \(10^{6} Pa\)
    \item steel: \( 10^{11} N/m^2 = 10^{11} Pa\)
    \item rubber: \( 10^7 - 10^8 N/m^2\)
\end{itemize}
\vspace{-0.7em}
\notes{
  Young's modulus of steel \\
  link: https://www.sciencedirect.com/science/article/pii/S0924013615301011 \\
  Young's modulus of human bone \\
  https://onlinelibrary.wiley.com/doi/epdf/10.1002/jor.1100080416
}

\subsubtitle{material vs linear material:}
\lis{
  \item co-rotational linear material: simplest possible choice for non-linear material
  \item 2nd popular material: St.Venant-Kirchhoff (StVK) material
}

\subsubtitle{Possion's ratio (nu):}
\equ{\frac{\Delta v}{v} = (1-2v) \cdot \frac{\Delta l}{l}}
\lis{
  \item measure volume preservation when deform
  \item when \(v=\frac{1}{2}\) volume preserved
  \item when \(v=0\) no shrink in width
  \item \(-1<=v<=\frac{1}{2}\)
  \item in general, volume preserved in real life
  \item human body preserve volume becuase \(70\%\) of human is made of water and water preserve volume
}
\subsubtitle{Linear FEM vs. FEM}
\lis{
  \item Linear FEM: deform more, inprecise
}

\subsubtitle{Given a paragraph, fill out the terms, put in Young's modulus and Energy in here}
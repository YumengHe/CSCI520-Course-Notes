\subtitle{7. Motion Capture}

\subsubtitle{What is Motion Capture?}
\lis{
  \item performance animation
  \item Motion capture is the process of tracking real-life motion in 3D and recording it for use in any number of applications.
}

\subsubtitle{Why Motion Capture?}
\lis{
  \item Keyframes are generated by instruments measuring a human performer - they do not need to be set manually
  \item The details of human motion such as style, mood, and shifts of weight are reproduced with little effort
  \item Keyframes are slow
}

\subsubtitle{Why not?}
\lis{
  \item Difficult for non-human characters: Can you move like a hamster/duck/eagle? Can you capture a hamster's motion?
  \item Actors needed: Which is more economical: Paying an animator to place keys or hiring a martial arts expert
}

\subsubtitle{When to use Motion Capture?}
\lis{
  \item Complicated character moation: Where "uncomplicated" ends and "complicated" begins is up to question; A walk cycle is often more easily done by hand; A Flying Monkey Kick might be worth the overhead of mocap
  \item Can an actor better express character personality than the animator?
}

\subsubtitle{1. Mocap Technologies: Optical (Camera)}
\lis{
  \item Multiple high-resolution, high-speed cameras
  \item Light bounced from camera off of reflective markers
  \item High quality data
  \item Markers placeable anywhere
  \item Lost of work to extract joint angles
  \item Occlusion
  \item Which marker is which? (correspondence problem)
  \item 120-240 Hz @ 1Megapixel
  \item equipments are expensive
  \item markers are on the surface
  \item need manual cleanup to get good data
  \item highest resulting
  \item most common in film
  \item \name{CMU Motion Capture Dataset}
  \item Facial Motion Capture
}

\subsubtitle{2. Mocap Technologies: Electromagnetic}
\lis{
  \item Sensors give both position and orientation
  \item No occlusion or correspondence problem
  \item Little post-processing
  \item Limited accuracy
  \item No camera
  \item Con: very heavy backpack
  \item \name{NOTIOM}: chinese company, design a network s.t. can transit data to a computer that does not need to be in the backpack.
}

\subsubtitle{3. Mocap Technoligies: Exoskeleton}
\lis{
  \item Really Fast (~500Hz)
  \item No occlusion or correspondence problem
  \item Little error
  \item Movement restricted
  \item Fixed sensors
  \item Con: alter the motion due to equipment
}

\subsubtitle{Question: How can we determine if 2 pose are similar?}
\lis{
  \item $P_1, P_2 \in \mathbb{R}^{60}$ position
  \item $\dot{P_1}, \dot{P_2} \in \mathbb{R}^{60}$ velocity 
  \item Loss = $\sum_{i=1}^{60}w^i(P_1^i-P_2^i)^2 + \lambda \cdot \sum_{i=1}^{60}\hat{w^i}(\dot{P_1}^i-\dot{P_2}^i)^2, \, w^i \in \mathbb{R}$
  \item Edge $\Leftrightarrow$ Loss $< \varepsilon$
  \item \name{J. Lee, J. Chai, P. Reitsma, J. Hodgins, N. Pollard: Interactive Control of Avatars Animated with Human Motion Data, SIGGRAPH 2002}
}
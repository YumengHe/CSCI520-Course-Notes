\topic{19. CUDA}

\subsubtitle{What is GPGPU?}
\lis{
  \item General-Purpose computing on a Graphics Processing Unit 
  \item Using graphic hardware for non-graphic computations
}

\subsubtitle{What is CUDA?}
\lis{
  \item Compute Unified Device Architecture
  \item Parallel computing platform and API by Nvidia
  \item Compute Unified Device Architecture
  \item Software architecture for managing data-parallel programming
  \item Introduced in 2007; still actively updated
  \item Con: collision detection is slow because of branching
}

\subsubtitle{Motivation}
\lis{
  \item GPU explode when image size increase
}

\subsubtitle{CPU vs. GPU}
\lis{
  \item CPU: Fast caches, Branching adaptability, High performance
  \item GPU: Multiple ALUs, Fast onboard memory, High throughput on parallel tasks (Executes program on each fragment/vertex)
  \item CPUs are great for task parallelism
  \item GPUs are great for data parallelism 
  \item Hardware - GPU has more transistors devoted to data processing 
}

\subsubtitle{GPU Memory Architecture}
\lis{
  \item Uncached:
  \item Registers (fastest)
  \item Shared Memory
  \item Local Memory
  \item Global Memory (fourth fastest)
  \item Cached: Constant Memory; Texture Memory
}

\subsubtitle{Software Requirements / Tools}
\lis{
  \item CUDA device driver (download required)
  \item CUDA Toolkit (compiler, CUBLAS, CUFFT)
  \item \name{nvcc} (NVIDIA's CUDA Compiler) compiles .cu and g++ compile .c .cpp files and connect
  \item CUDA Software Development Kit (Emulator)
}

\subsubtitle{To compute, we need to:}
\lis{
  \item Allocate memory for the computation on the GPU (incl. variables)
  \item Provide input data
  \item Specify the computation to be performed
  \item Read the results from the GPU(output)
  \item While CUDA computing on GPU, CPU will not wait by default, eg. running user interface, mouse clicking, or any other program
}

\subsubtitle{Process}
\lis{
  \item Allocate Memory in the GPU card
  \item Copy content from the host’s memory to the GPU card memory
  \item Execute code on the GPU
  \item Copy results back to the host memory
}

\subsubtitle{The Kernel}

\subsubtitle{Grid and Block Size}
block = 3 warp 
warp = 32 threads
multiple-processor has limited threads can perform

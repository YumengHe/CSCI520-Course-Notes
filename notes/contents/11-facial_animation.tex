\topic{11. Physically-based Facial Modeling}

\subsubtitle{Motivation}
\lis{
  \item Why a talking head? \\
        Enhanced comminication for pople with disabilities \\
        Training scenario software \\
        Entertainment: Games and Movies
  \item Why physically based? \\
        Unburdens animators \\
        Provides more realistic looking simulations 
}

\subsubtitle{Anatomy of the face}
\lis{
  \item There are 268 voluntary muscles that contribute to your expression 
  \item Linear/Parallel muscles (share a common anchor): contract longitudinally towards their origin, e.g. levator labii sup., zygomaticus minor/major 
  \item Sheet muscles (run parallel, activated together): composed of several linear muscles side-by-side, e.g. frontalis
  \item Sphincter muscles (contract to a center point): contract radially towards a center point, e.g. orbicularis oris, orbicularis oculi
}

\subsubtitle{Muscles}
\lis{
  \item Bundles of thousands of individual fibers: Thankfully, can be modeled as these bundles; When activated, all of the fibers contract 
  \item Contraction only: Most parts of the body use opposing pairs of muscles, but the face relies on the skin 
  \item Bulging: Occurs due to volume preservation; Thicker on contraction, thinner on elongation; Important for realistic faces (e.g. pouting lips)
}

\subsubtitle{Skin}
\lis{
  \item Epidermis: Thin, stiff layer of dead skin (no real simulation)
  \item Dermis: Primary mechanical layer; Collagen and Elastin fibers 
  \item Subcutaneous or Fatty tissue (Fat): Allows skin to slide over muscle bundles; Varies in thickness
}

\subsubtitle{Modeling viscoelastic skin}
\lis{
  \item Collagen fibers - low strain for low extensions 
  \item Near maximum expansion, stress rises quickly 
  \item When allowed to, elastin fibers return system to rest state quickly 
  \item Biphasic model: Two picewise linear models; Threshold extension to pock spring constant 
}

\subsubtitle{The skull}
\lis{
  \item Unlike most of the body, the face only has a sigle joint and 2 bones: jaw bone and skull  
  \item All other expression is due to computer unfriendly soft tissues 
  \item Can be treated as a rigid body
}

\subtitle{Facial Action Coding System (FACS)} Encode the expression using "atomic" expressions (as defined by the FACS standard). Specify intensity A-E, and permit combinations of expressions. Example: 26E + 11F + 5B.
\lis{
  \item Proposed by Ekman and Friesan in 1978 
  \item Describes facial movement in terms of the muscles involved 
  \item Purposely ignores invisible and non movement changes (such as blushing)
  \item Defines 46 action units pertaining to expression-related muscles 
  \item Additional 20 action units for gross head movement and eye gaze
  \item qualitative
}

\subsubtitle{Blendshapes}: linear blendshape facial rig
\lis{
  \item input FACS coordinate, output facial position 
  \item all the mesh must have the same amount of vertices, only position changes 
  \item model: \equ{x = n + \sum_{i=1}^{N}\alpha_i (b_i - n)}
  \item $1 \geq \alpha_i \geq 0$: precreated when design the game, "sliders", "extra parameters"
  \item $b_i-n$: difference between blendshape and neutral 
  \item \name{ICP iterative closet point [USC 1991 Gerard Medioni]}
}

\subsubtitle{Combination shapes}: improve realism 
\lis{
  \item $c_{ij}, i = 1, ..., N, \, j= 1,..., N$
  \item $\alpha_4=1, \, \alpha_7 = 1 \Rightarrow \beta_{47}=1 $
  \item $x=n+\sum_{i=1}^{N}\alpha_i(b_1-n)+\sum_{i<j}\beta_{ij}(c_{ij}-n)$
}

\subtitle{MPEG-4 Facial Animation} Specify the expression by giving the displacement vector for a set of facial landmarks. The landmarks are specified by the standard.
\lis{
  \item Defines 84 feature points with position and zone of influence on a few basis keyframes of a standard 3D mesh 
  \item Defines animation independently of the visual rep. 
  \item 68 facial action parameters (FAPS), defined in terms of face independedn FAP units (FAPUs)
  \item Most define a rotation or translation of one or more feature points, with a few selecting entirely new key frames (e.g. an emotion basis)
  \item Same animation can be used on different model, provided the model is properly annotated 
  \item quantitative 
  \item specify physical displacement of landmark
  \item pro: can mock up the point, render easily 
  \item con: hard to simulate 
}
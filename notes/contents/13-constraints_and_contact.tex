\topic{13. Constraints and contact}
The idea of constrained particle dynamics is that our description of the system includes not only
particles and forces, but restrictions on the way the particles are permitted to move. For example,
we might constrain a particle to move along a specified curve, or require two particles to remain
a specified distance apart. The problem of constrained dynamics is to make the particles obey
Newton’s laws, and at the same time obey the geometric constraints.